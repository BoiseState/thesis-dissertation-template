% The \documentclass command is the first command in a LaTeX file.
%\documentclass{bsu-cs}  % default is thesis
%\documentclass[project]{bsu-cs}  % for project reports
%\documentclass[dissertation]{bsu-cs}  % for dissertation
% Comment the above line and uncomment the following two lines to produce a twosided document, suitable for printing.
\documentclass[dissertation, twoside]{bsu-cs}  % for dissertation
\usepackage[inner=1.5in,outer=1in,letterpaper,twoside]{geometry}

% "bsu-cs" is our "in-house" class file for Boise State dissertations, theses
% and project reports.

%%%%%%%%%%%%%%%%%%%%%%%%%%%%%%%%%%%%%%%%%%%%%%%%%%%%%%%%%%%%%%%%%%%%%%%%%%%%%%
% Some General Information about LaTeX
%
% LaTeX source (also called "code") files are split into two basic parts:
% the preamble, which contains package loading commands, and other
% document-wide definitions, and the main document text.  The preamble starts
% with the \documentclass command and ends with the \begin{document} command.
% The main text is contained between the \begin{document} and \end{document}
% commands.
%
% Only LaTeX commands (and whitespace) are allowed in the preamble.  A LaTeX
% command has the form
%
%   \<name>{<arg>}{<arg>}...
%
% where the number of arguments is set by the command definition.
% Some commands allow an optional argument
%
%   \name[<opt>]{<arg>}{<arg>}...
%
% which may or may not be given.  There are also LaTeX environments
%
%   \begin{<name>}...\end{<name>}
%
% which can be nested, but must be balanced.
%
% (If you haven't already guessed, anything after a '%' character on a line
% is taken as a comment and ignored.)
%
% In the main text all non-commands and non-comments are treated as ordinary
% text.  LaTeX normally treats any run of spaces as a single space.  Line
% breaks are ignored (TeX uses its own linebreaking algorithm); blank lines
% are paragraph separators.  Runs of blank lines are treated as a single
% blank line.  A common mistake novice users make is to try to use literal
% spacing and line breaks in the .tex file to control formatting.

%%%%%%%%%%%%%%%%%%%%%%%%%%%%%%%%%%%%%%%%%%%%%%%%%%%%%%%%%%%%%%%%%%%%%%%%%%%%%%
% Packages

% The 'graphicx' package is quite extensive; the main purpose here is
% the includion of PostScript graphics.  The 'dvips' option makes it work
% better with 'dvips' (which is what is often used under Linux)
%\usepackage[dvips]{graphicx}
% Comment the above line and uncomment below to use pdflatex
\usepackage[pdftex]{graphicx}
% Then the figures should be in pdf, jpg or png format.

% Others to consider
\usepackage{listings}
\usepackage{color}
\definecolor{listinggray}{gray}{0.95}
\definecolor{mygreen}{rgb}{0,0.6,0}
\definecolor{mygray}{rgb}{0.5,0.5,0.5}
\definecolor{mymauve}{rgb}{0.58,0,0.82}
\definecolor{maroon}{rgb}{0.5,0.00,0.0}


\lstdefinestyle{javanum}{
  backgroundcolor=\color{listinggray},   % choose the background color; you must add \usepackage{color} or \usepackage{xcolor}
  basicstyle=\footnotesize\ttfamily, % the size of the fonts that are used for the code
  breakatwhitespace=false,         % sets if automatic breaks should only happen at whitespace
  breaklines=true,                 % sets automatic line breaking
  captionpos=none,                   % sets the caption-position to bottom
  aboveskip=\smallskipamount,
  belowskip=\smallskipamount,     % default is \medskipamount
  commentstyle=\color{mygreen},    % comment style
  deletekeywords={...},            % if you want to delete keywords from the given language
  escapeinside={\%*}{*)},          % if you want to add LaTeX within your code
  extendedchars=true,              % lets you use non-ASCII characters; for 8-bits encodings only, does not work with UTF-8
  frame=none,                      % adds a frame around the code (none, single)
  keepspaces=true,                 % keeps spaces in text, useful for keeping indentation of code (possibly needs columns=flexible)
  keywordstyle=\color{maroon},       % keyword style
  language=Java,                   % the language of the code
  morekeywords={*,...},            % if you want to add more keywords to the set
  numbers=left,                    % where to put the line-numbers; possible values are (none, left, right)
  numbersep=5pt,                   % how far the line-numbers are from the code
  numberstyle=\tiny\color{mygray}, % the style that is used for the line-numbers
  rulecolor=\color{black},         % if not set, the frame-color may be changed on line-breaks within not-black text (e.g. comments (green here))
  showspaces=false,                % show spaces everywhere adding particular underscores; it overrides 'showstringspaces'
  showstringspaces=false,          % underline spaces within strings only
  showtabs=false,                  % show tabs within strings adding particular underscores
  stepnumber=1,                    % the step between two line-numbers. If it's 1, each line will be numbered
  stringstyle=\color{mymauve},     % string literal style
  tabsize=4,                       % sets default tabsize to 2 spaces
%   title=\lstname                   % show the filename of files included with \lstinputlisting; also try caption instead of title
}

\lstdefinestyle{javanonum}{
  backgroundcolor=\color{listinggray},   % choose the background color; you must add \usepackage{color} or \usepackage{xcolor}
  basicstyle=\footnotesize\ttfamily, % the size of the fonts that are used for the code
  breakatwhitespace=false,         % sets if automatic breaks should only happen at whitespace
  breaklines=true,                 % sets automatic line breaking
  captionpos=none,                 % sets the caption-position to bottom
  aboveskip=\smallskipamount,
  belowskip=\smallskipamount,      % default is \medskipamount
  commentstyle=\color{mygreen},    % comment style
  deletekeywords={...},            % if you want to delete keywords from the given language
  escapeinside={\%*}{*)},          % if you want to add LaTeX within your code
  extendedchars=true,              % lets you use non-ASCII characters; for 8-bits encodings only, does not work with UTF-8
  frame=none,                      % adds a frame around the code (none, single)
  keepspaces=true,                 % keeps spaces in text, useful for keeping indentation of code (possibly needs columns=flexible)
  keywordstyle=\color{maroon},       % keyword style
  language=Java,                   % the language of the code
  morekeywords={*,...},            % if you want to add more keywords to the set
  numbers=none,                    % where to put the line-numbers; possible values are (none, left, right)
  rulecolor=\color{black},         % if not set, the frame-color may be changed on line-breaks within not-black text (e.g. comments (green here))
  showspaces=false,                % show spaces everywhere adding particular underscores; it overrides 'showstringspaces'
  showstringspaces=false,          % underline spaces within strings only
  showtabs=false,                  % show tabs within strings adding particular underscores
  stringstyle=\color{mymauve},     % string literal style
  tabsize=4,                       % sets default tabsize to 4 spaces
% title=\lstname                   % show the filename of files included with \lstinputlisting; also try caption instead of title
}
\lstset{style=javanonum}


% The 'subfigure' package does a nice job handling and labelling subfigures
%\usepackage{subfig}

% If you want to use PostScript fonts (or the Nimbus knockoffs) try
%\usepackage{mathptmx}
%\usepackage[scaled=0.92]{helvet}

% The 'citesort' package puts mutliple citation numbers in order
%\usepackage{citesort}


% Generate intra-document links and also allows url and href commands for
% to generate embedded links in the document.
\usepackage{hyperref}
\hypersetup{
 	colorlinks=true,
 	allcolors=blue,
}



%%%%%%%%%%%%%%%%%%%%%%%%%%%%%%%%%%%%%%%%%%%%%%%%%%%%%%%%%%%%%%%%%%%%%%%%%%%%%%
% Local commands

% Define your macros (new commands) here.


%%%%%%%%%%%%%%%%%%%%%%%%%%%%%%%%%%%%%%%%%%%%%%%%%%%%%%%%%%%%%%%%%%%%%%%%%%%%%%
% Front Matter Definitions
%%%%%%%%%%%%%%%%%%%%%%%%%%%%%%%%%%%%%%%%%%%%%%%%%%%%%%%%%%%%%%%%%%%%%%%%%%%%%%

% These definitions are used by the commands below to construct the front
% matter pages.

% Document title
% The \titleBreak command produces a line break in the title on the
% title page (this may be necessary to keep longer titles from looking
% strange)
\title{An Ode to Wabi-sabi:\titleBreak Homage to a Great and Wonderful Person}

% Author's name (must be exactly what name the Registrar has)
% normally it is of the form "First Middle Last"
\author{Wabi-sabi Admirer}

% Day of oral defense
\defenseDate{1st December, 2022}

% Month of graduation (do not abbreviate)
\graduationMonth{December}

% Year of graduation
\graduationYear{2022}

% Advisor (chair)
% Use full name, with surname last, without any titles
% The \advisor command accepts an optional argument that specifies the
% advisor's position, which defaults to "Advisor"; for example,
%
% \advisor[Chair]{Wabi-sabi the Great}
%
% causes "Chair" to appear on the approval page (the one with the signatures)
% in place of "Advisor"
\advisor{Wabi-sabi The Great, Ph.D.} % also the chair of your committee

% Committee members: \committeeA is listed first after the advisor,
% \committeeB second, etc.  An optional argument is accepted to replace
% the default "Committee Member" position.
\committeeA{Aaa, Ph.D.}
\committeeB{Bbb, Ph.D.}

% Masters or PhD committees normally have three members, but in case there is
% a need for more, use these commands
%\committeeC[Ex-Officio Committee Member]{ccc}
%\committeeD{ddd}


% Full name of the degree
%\degree{Master of Science}
\degree{Doctor of Philosophy}


% Full name of major. For PhD in Computing, add the appropriate emphasis in parentheses.
%\major{Computer Science}
\major{Computing (Emphasis)}

% Major department. These aren't really used.
\department{Computer Science}

% College. These aren't really used.
\college{Engineering}

% Current department chair (at the moment, this is not used).
\departmentChair{Jerry Fails}


% Graduate coordinator, only used for masters.
\gradCoordinator{Francesca Spezzano}


% Document abstract
\abstract{ An \emph{abstract} is a brief summary of the document. A typical
  abstract provides a brief introduction, enough to provide context for the document, explains
  the purpose of the thesis or dissertation or project, and summarizes the major results and
  conclusions.  Keep in mind that a casual observer is likely to judge the content of the document
  by the abstract and title alone.  (There is an old adage: ``in a joke, the punchline comes
  at the end; in a paper [or thesis], it comes in the abstract.'')  A single concise paragraph
  usually suffices for the abstract.  If it spills onto a second page, it is probably too long.
}


% The remaining elements are optional

% \includeCopyright causes a copyright page to be produced.  It is not
% required; in fact, a copyright notice is no longer needed to ensure
% copyright protection.
\includeCopyright

% \maxPage is used to format the page numbers in the table of contents.
% The actual value is not important: the width of the value given is used
% as the maximum width of any page number.  The following values are suggested
%
% \maxPage{99} for less than 100 pages (this is the default)
% \maxPage{199} for less than 200 pages
% \maxPage{999} for less than 1000 pages
%
% (If the work is more than 1000 pages, consider cutting it down!)
% The space has to be wide enough for front matter page numbers, which are
% miniscule Roman numerals.  This might have to be widened if there are
% more than 12 front matter pages (which might happen if there is a long
% list of symbols or list of abbreviations, like this template has).
\maxPage{199}

% The Acknowledgments page is a place for the author to express thanks
% or generally acknowledge anyone or anything that assisted in the
% project or research.  If the student was supported even in part by
% funding from research grant, it needs to be acknowledged here.
% The Acknowledgments section can have multiple pages, just like the
% Abstract section, but like the abstract, more than one page is probably
% too long.  Acknowledgments are often written in the third person, but
% it is one place (the only place) in the document where first-person
% references (i.e., the use of "I") can be acceptable.
\acknowledgments{
The author wishes to express gratitude to Wabi-sabi.  This work would have been
partially supported by some particular grant, if there was one.
}

% The work can be "dedicated" to someone, typically family members or
% someone similar.  Unlike acknowledgments, which are included in almost
% every thesis or dissertation or project, many students omit the dedication.
\dedication{dedicated to Wabi-sabi}


% A brief autobiographical sketch can be included.  Many students omit this.
% If it is included, remember that it is a sketch, so keep it brief.  Also
% try to limit the content to academically relevant material, such as under-
% graduate work, previous graduate work, employment, etc.  Some students
% like to include their military service.  Otherwise, unless you have
% overcome some particular adversity, it's probably better to leave out
% personal information.
\biosketch{
Wabi-sabi Admirer was born admiring Wabi-sabi. Wabi-sabi Admirer has been tinkering
with admiration of Wabi-sabi for a long time. Now it is time to be blessed by
Wabi-sabi.
}

% The title of the bibliography defaults to "References" but if you include
% references that are not specifically cited in the text, it needs to be
% titled "Bibliography" instead.
%\bibliographyName{Bibliography}


% (End of preamble)
%%%%%%%%%%%%%%%%%%%%%%%%%%%%%%%%%%%%%%%%%%%%%%%%%%%%%%%%%%%%%%%%%%%%%%%%%%%%%%



% \begin{document} begins the document text, including front matter material

\begin{document}

% The \frontmatter command prepares for front matter material
\frontmatter  %! Do not remove!

% The \buildFrontPages builds all the front matter pages
\buildFrontPages %! Do not remove!

% The (optional) list of abbreviations goes here
\begin{listAbbreviations}
  \item[LOL] Laughing Out Loud
  \item[OMG] Oh My God!
  \item[ROFL] Rolling on the Floor Laughing
\end{listAbbreviations}


% The (optional) list of symbols goes here
\begin{listSymbols}
  \item[$\sqrt{2}$] square root of 2
  \item[$\lambda$] lambda symbol, normally used in lambda calculus but
    it sometimes gets used for wavelength as well
\end{listSymbols}


% The main text of the work starts here.  The first chapter is normally
% named "Introduction".

\mainmatter

% Note that judiciously placed comments can serve as landmarks, just as
% in program source code.

%%%%%%%%%%%%%%%%%%%%%%%%%%%%%%%%%%%%%%%%%%%%%%%%%%%%%%%%%%%%%%%%%%%%%%%%%%%%%%
%
% Chapter: Introduction
%
%%%%%%%%%%%%%%%%%%%%%%%%%%%%%%%%%%%%%%%%%%%%%%%%%%%%%%%%%%%%%%%%%%%%%%%%%%%%%%

\chapter{Introduction}

% The \label{<tag>} command allows the chapter to be referenced elsewhere
% in the text by the <tag>: 'Chapter~\ref{ch:intro}' gets formatted as
% "Chapter 1" or whatever number this chapter gets assigned.  It usually
% requires a second 'latex' execution to get the labels right.  (The labels
% are stored in the ".aux" file, and that gets updated each time latex runs.)
%   Writers who use LaTeX a lot are often in the habit of adding labels to
% chapter, figure, and table, then adding other labels to things like sections
% and subsections as they are needed.  Label tags follow no specific format,
% but it's typical to include a colon-terminated prefix to indicate what
% is being labeled, e.g.
%
%   ch:<name>  for chapters
%   sec:<name> for sections
%   tbl:<name> for tables
%   fig:<name> for figures
%
% WARNING: Do not use numbers or special characters such as '_', '$', etc.
%          in label tags (although ':' is okay).

\label{ch:intro}

% \section adds a first-level subheading
\section{What is this?}
This is a template that allows you to typeset your dissertation (or thesis or project report)
in the format approved by the Boise State Graduate College. It greatly reduces your typesetting
work and helps you produce an aesthetically pleasing and consistent document. The template
depends on a style class, titled \texttt{bsu-cs.cls} that was specifically designed for Boise
State Computer Science students but would also work for students from other departments.

\subsection{Where are the style class files?}

% \cite{<tag>} is bibilography citation; <tag> must match a bibliography
% item label.  It can take multiple, comma-separated, arguments.

% The '~' is a "tie", which is an unbreakable space.  That is, it is
% the same a a space but TeX won't break the line there.  Ties are used
% between text and a reference or citation, to avoid something like Figure
% 1, which looks weird.

Please consult the guide from Graduate College~\cite{guide} for resolving any style issues
that are not addressed by the style class files that are provided along with this document. The
files associated with this style can be found on the GitHub website~\cite{mainrepo}.

% \texttt{} changes to "typewriter" (monospaced) font, which is typically
% used for literal code or filenames.
% \emph{} "emphasizes" the text, which normally means placing it in
% italics if the running text is in Roman font, or vice-versa.

The file \texttt{bsu-cs.cls} contains the formatting directives for the \emph{bsu-cs} style.
It is based on the standard \LaTeX\ \texttt{report} style with 12 point font option.

% \begin{enumerate} creates a numbered list.
\begin{enumerate}
% \item starts a new list item.
\item Simply copy \texttt{bsu-cs.cls} to the directory containing your \LaTeX\ document.
That way, \LaTeX\ will find it, because it looks in current directory by default.

\item Upload the \texttt{bsu-cs.cls} file along with this template (and associated files)
to Overleaf to use web-based LaTeX.

\item The current style class file may be installed in some directory available system wide.  (Ask your
systems administrator if this is the case).  You will have to include that directory in the path
\LaTeX\ uses to search for input files.  Under Linux, this is controlled by the \texttt{TEXINPUTS}
environment variable, which can be set in the \texttt{.bashrc} file in your home directory.
For example
% The 'verbatim' environment formats text in the typewriter font,
% just as it appears in the file.
%
% WARNING: 'verbatim' doesn't expand tab characters, so if you cut and paste
% be sure to remove any tabs (you can use the 'M-x untabify' command in emacs).

\begin{verbatim}
    TEXINPUTS=.:/usr/local/texinputs/:/usr/share/texmf//
    export TEXINPUTS
\end{verbatim}

adds \texttt{/usr/local/texinputs}, a possible location for \texttt{bsu-cs.cls}, although it
will not take effect until you \texttt{source} the \texttt{.bashrc} file, or log in again.

\item Install the style class files in a directory under your accounts and set the \texttt{TEXINPUTS}
variable accordingly.

\end{enumerate}

% If you have blank lines before the \begin{enumerate} or after the
% \end{enumerate} it makes the LaTeX code easier to read, but it also
% starts a new paragrah.  You can always use empty comments for readablilty.
%
% Although, you can always use \noindent to prevent paragraph indentation.

The first or second way is recommended, because they involve making a local copy of the style
file.  This assures your document format will not be affected affected by subsequent updates
to \texttt{bsu-cs.cls} (but gives you the option to copy the updated file if you want).

\section{Get ready for Wabi-sabi}\label{sec:getReady}

% The \footnote command creates a footnote
So who is Wabi-sabi? We need a lot of text in here to see what happens when we hit the bottom of
a page with text and try out things like footnotes
\footnote{What's not to like about footnotes, anyway?  Brian O'Nolan and George MacDonald Fraser both used them to great effect}.
So here is some extra stuff:%
\footnote{Too many footnotes, however, can be distracting.}
% A '%' comment character at the end of a line, like above, has the
% effect of "escaping" the newline, thereby avoiding adding a space.
% In this case, it prevents there being space between the end of the
% text and the footnote superscript

stuff stuff stuff stuff stuff stuff stuff stuff stuff stuff stuff stuff stuff
stuff stuff stuff stuff stuff stuff stuff stuff stuff stuff stuff stuff stuff
stuff stuff stuff stuff stuff stuff stuff stuff stuff stuff stuff stuff stuff
stuff stuff stuff stuff stuff stuff stuff stuff stuff stuff stuff stuff stuff
stuff stuff stuff stuff stuff stuff stuff stuff stuff stuff stuff stuff stuff
stuff stuff stuff stuff stuff stuff stuff stuff stuff stuff stuff stuff stuff
stuff stuff stuff stuff stuff stuff stuff stuff stuff stuff stuff stuff stuff
stuff stuff stuff stuff stuff stuff stuff stuff stuff stuff stuff stuff stuff
stuff stuff stuff stuff stuff stuff stuff stuff stuff stuff stuff stuff stuff
stuff stuff stuff stuff stuff stuff stuff stuff stuff stuff stuff stuff stuff
stuff stuff stuff stuff stuff stuff stuff stuff stuff stuff stuff stuff stuff
stuff stuff stuff stuff stuff stuff stuff stuff stuff stuff stuff stuff stuff
stuff stuff stuff stuff stuff stuff stuff stuff stuff stuff stuff stuff stuff
stuff stuff stuff stuff stuff stuff stuff stuff stuff stuff stuff stuff stuff
stuff stuff stuff stuff stuff stuff stuff stuff stuff stuff stuff stuff stuff
stuff stuff stuff stuff stuff stuff stuff stuff stuff stuff stuff stuff stuff

We need lots more stuff to get a full page of text, without a chapter or section heading,
so we can check all the margins.

stuff stuff stuff stuff stuff stuff stuff stuff stuff stuff stuff stuff stuff
stuff stuff stuff stuff stuff stuff stuff stuff stuff stuff stuff stuff stuff
stuff stuff stuff stuff stuff stuff stuff stuff stuff stuff stuff stuff stuff
stuff stuff stuff stuff stuff stuff stuff stuff stuff stuff stuff stuff stuff
stuff stuff stuff stuff stuff stuff stuff stuff stuff stuff stuff stuff stuff
stuff stuff stuff stuff stuff stuff stuff stuff stuff stuff stuff stuff stuff
stuff stuff stuff stuff stuff stuff stuff stuff stuff stuff stuff stuff stuff
stuff stuff stuff stuff stuff stuff stuff stuff stuff stuff stuff stuff stuff
stuff stuff stuff stuff stuff stuff stuff stuff stuff stuff stuff stuff stuff
stuff stuff stuff stuff stuff stuff stuff stuff stuff stuff stuff stuff stuff
stuff stuff stuff stuff stuff stuff stuff stuff stuff stuff stuff stuff stuff
stuff stuff stuff stuff stuff stuff stuff stuff stuff stuff stuff stuff stuff
stuff stuff stuff stuff stuff stuff stuff stuff stuff stuff stuff stuff stuff
stuff stuff stuff stuff stuff stuff stuff stuff stuff stuff stuff stuff stuff
stuff stuff stuff stuff stuff stuff stuff stuff stuff stuff stuff stuff stuff
stuff stuff stuff stuff stuff stuff stuff stuff stuff stuff stuff stuff stuff
stuff stuff stuff stuff stuff stuff stuff stuff stuff stuff stuff stuff stuff
stuff stuff stuff stuff stuff stuff stuff stuff stuff stuff stuff stuff stuff
stuff stuff stuff stuff stuff stuff stuff stuff stuff stuff stuff stuff stuff
stuff stuff stuff stuff stuff stuff stuff stuff stuff stuff stuff stuff stuff
stuff stuff stuff stuff stuff stuff stuff stuff stuff stuff stuff stuff stuff
stuff stuff stuff stuff stuff stuff stuff stuff stuff stuff stuff stuff stuff
stuff stuff stuff stuff stuff stuff stuff stuff stuff stuff stuff stuff stuff
stuff stuff stuff stuff stuff stuff stuff stuff stuff stuff stuff stuff stuff
stuff stuff stuff stuff stuff stuff stuff stuff stuff stuff stuff stuff stuff
stuff stuff stuff stuff stuff stuff stuff stuff stuff stuff stuff stuff stuff
stuff stuff stuff stuff stuff stuff stuff stuff stuff stuff stuff stuff stuff
stuff stuff stuff stuff stuff stuff stuff stuff stuff stuff stuff stuff stuff
stuff stuff stuff stuff stuff stuff stuff stuff stuff stuff stuff stuff stuff
stuff stuff stuff stuff stuff stuff stuff stuff stuff stuff stuff stuff stuff
stuff stuff stuff stuff stuff stuff stuff stuff stuff stuff stuff stuff stuff
stuff stuff stuff stuff stuff stuff stuff stuff stuff stuff stuff stuff stuff
stuff stuff stuff stuff stuff stuff stuff stuff stuff stuff stuff stuff stuff
stuff stuff stuff stuff stuff stuff stuff stuff stuff stuff stuff stuff stuff
stuff stuff stuff stuff stuff stuff stuff stuff stuff stuff stuff stuff stuff
stuff stuff stuff stuff stuff stuff stuff stuff stuff stuff stuff stuff stuff
stuff stuff stuff stuff stuff stuff stuff stuff stuff stuff stuff stuff stuff
stuff stuff stuff stuff stuff stuff stuff stuff stuff stuff stuff stuff stuff
stuff stuff stuff stuff stuff stuff stuff stuff stuff stuff stuff stuff stuff
stuff stuff stuff stuff stuff stuff stuff stuff stuff stuff stuff stuff stuff
stuff stuff stuff stuff stuff stuff stuff stuff stuff stuff stuff stuff stuff
stuff stuff stuff stuff stuff stuff stuff stuff stuff stuff stuff stuff stuff
stuff stuff stuff stuff stuff stuff stuff stuff stuff stuff stuff stuff stuff
stuff stuff stuff stuff stuff stuff stuff stuff stuff stuff stuff stuff stuff
stuff stuff stuff stuff stuff stuff stuff stuff stuff stuff stuff stuff stuff
stuff stuff stuff stuff stuff stuff stuff stuff stuff stuff stuff stuff stuff
stuff stuff stuff stuff stuff stuff stuff stuff stuff stuff stuff stuff stuff
stuff stuff stuff stuff stuff stuff stuff stuff stuff stuff stuff stuff stuff

%%%%%%%%%%%%%%%%%%%%%%%%%%%%%%%%%%%%%%%%%%%%%%%%%%%%%%%%%%%%%%%%%%%%%%%%%%%%%%
%
% Chapter: The Greatness of Wabi-sabi
%
%%%%%%%%%%%%%%%%%%%%%%%%%%%%%%%%%%%%%%%%%%%%%%%%%%%%%%%%%%%%%%%%%%%%%%%%%%%%%%

\chapter{The Greatness of Wabi-sabi}\label{ch:wabisabiGreatness}

\section{Previous work}
The greatness of Wabi-sabi The Great derives from her early work as documented in
her books~\cite{ws:book1,ws:book2}.

\section{What are their names?}
Please consult the articles by Admirer~\cite{me:paper1} and Admirer, Smith and Doe~\cite{me:paper2}
for more details. Note that the references are cited by the last names of all authors for three
authors or less. For more than three authors, ``et al'' can be used.

%! The \label command applies to the most recent heading in the main text,
%! or the current figure or table.
%! The \pageref{<tag>} command references the page number of a label.
Check the References on page~\pageref{references} for an example of how to format the references.

\section{The code of Wabi-sabi}

% \textbf{} sets text in bold face.  It is useful in headings, but
% is rarely used in running text.  It is used here to denote literal
% LaTeX commands, and to illustrate its use.  Normally it is preferable
% to use \emph{} to emphasize text.
When showing a program fragment, use the \textbf{verbatim} environment. However, when you wan
to show an algorithm, use either the \textbf{tabbing} or \textbf{algorithm} environment.

Thesis and dissertation text is normally ``double'' spaced.  It is customary to single-space
literal code.  Figure~\ref{fig:code} shows a sample Java program.

% The 'figure' environment starts a new figure, which is a type of
% "floating" element.  That means it goes where LaTeX decides to put it,
% rather than at the current point in the running text.  Don't be surprised
% when figures and tables get placed in unexpected places.
%
% You can control, to some degree, where floating elements go by setting
% the optional argument of \begin{figure}.  The options are
%
%   h - place the figure at this point, if possible (avoid this)
%   t - place the figure at the top of a page
%   b - place the figure at the bottom of a page
%   p - place the figure on a special page containing only figures and tables
%
% The options can be combined, and a '*' can be appended to make LaTeX
% try harder to match the request.  For example, \begin{figure}[tp*]
% allows the figure to be placed at the top of the page or on a special
% page ([tp*] is in fact the default set in "bsu-cs.cls").  Avoid using
% [h] by itself--it is inflexible and can cause formatting problems.  At
% the same time, it can be very useful in forcing figure placement.
% Finessing figure placement is a typesetting task that you just have to get
% used to, no matter what system you are using.
%
% WARNING: when LaTeX has trouble placing floating elements, it sometimes
%          puts all of them at the very end of the document.  So if your
%          figures and tables suddenly disappear, check the end of the
%          document.  Adding more 'p' specifiers can help with this.

\begin{figure}[t]

\begin{vcode}
% an alternative to 'singlespace', that also shrinks the
% typewriter font so that it blends better with the text
\begin{lstlisting}
/**
Compute x^n using recursive doubling technique. O(lg n) multiplications.
@param x  The base value, unlimited precision.
@param n  The exponent, an integer.
@return   The computed power as a BigInteger
*/
public static BigInteger power(BigInteger x, int n)
{
    BigInteger temp = x;
    BigInteger result = BigInteger.ONE;
	while (n != 0) {
        if ((n & 1) == 1)
            result = result.multiply(temp);
		if ((n = n >>>1) != 0)
           	temp = temp.multiply(temp);
    }
    return result;
}
\end{lstlisting}
\end{vcode}
%\caption{} gives the figure or table a caption, which is normally
% placed below the figure (but above a table).  The caption is also
% used as the title of the figure in the list of figures.  An optional
% argument gives the figure a "short title" for use in the list of
% figures, which is useful if the caption is long, or includes some
% extra information.
\caption[Repeated Squaring Power Method]{Repeated Squaring Power Method. This figure
also serves as an example of the inclusion of literal code.}
%
% WARNING: the labels can get screwed up if the label in a figure is given
%          before the caption (another one of LaTeX's quirks)
\label{fig:code}
\end{figure}


\section{Other mysteries of Wabi-sabi}

Here is an itemized list of all the mysteries of Wabi-sabi.
% The 'itemize' environment is for unordered lists: the items are bulleted
% rather than numbered.  A different bullet style is used for itemized
% lists inside itemized lists.
\begin{itemize}
\item Mystery 1.
\item Mystery 2.
\item Mystery 3.
\item Mystery 4.
\end{itemize}

% Tables are like figures, except that they are numbered separately
% and the caption is placed at the top.  Also there is a separate
% list of tables in the front matter.

Here is a simple table.

\begin{table}[ht] % 'h' in this case prevents the table from being placed
                 % too close to the figure
% By custom, the caption of a table is placed at the top of the table.
\caption{The Approximate Time of Parallelizing Each Code}
% (The label must follow the caption)
\label{table4}
\centering % tables need be horizontally centered
% The 'tabular' environment can be used to format tables.  Unfortunately
% it is one of the more cumbersome LaTeX constructs.
\begin{tabular}{|c|c|c|c|}\hline \hline
Parallel library/language  &WRS Code  &OCS Code  &ICSAMD Code\\ \hline
MPI                        &20 hours  &2 weeks   &1 month\\ \hline
HPF                        &3 hours   &1 1/2 weeks  &1 month\\ \hline
\end{tabular}
\end{table}


%%%%%%%%%%%%%%%%%%%%%%%%%%%%%%%%%%%%%%%%%%%%%%%%%%%%%%%%%%%%%%%%%%%%%%%%%%%%%%
%
% Chapter: not so great
%
%%%%%%%%%%%%%%%%%%%%%%%%%%%%%%%%%%%%%%%%%%%%%%%%%%%%%%%%%%%%%%%%%%%%%%%%%%%%%%

\chapter{The not so great things about Wabi-sabi}

\section{Figures}
Check Figure~\ref{fig:fuzzyImage} for what happens when Wabi-sabi gets compiled. This example
shows how to include an image (in PDF, JPG or PNG) into a LaTeX document.

\begin{figure}[ht]
% Code in figures is normally not centered, but graphical figures are
% (the \centering command works as well as the 'center' environment,
% because its scope is limited to the figure).
\begin{center}
\includegraphics*[width=4.0in,keepaspectratio]{figure}
\end{center}
\caption{How to Correct Errors in a Fuzzy Image}
\label{fig:fuzzyImage}
\end{figure}

\section{Tables}

Table~\ref{tbl:CSSSM} shows the formatting and labeling for a table.

\begin{table}[ht]
\caption{Complexity of Selection and Search in Sorted Matrices}
\label{tbl:CSSSM}
\centerline{
\begin{tabular}{|l|cccc|}
\hline
& Sorted $X+Y$ & \multicolumn{2}{c}{Matrix with sorted rows} & Matrix with sorted \\
&  & \multicolumn{2}{c}{and sorted columns} & columns \\ \cline{2-5}
& $|X|=|Y|=n$ &  $n \times m$, $m \leq n$ & $n \times n$ & $n \times m$ \\
\hline
$k = \Theta(mn)$ or $\Theta(n^2)$ & $\Theta(n)$ & $\Theta(m
\log (2n/m))$ & $\Theta(n)$ & $\Theta(m \log n)$ \\
\hline
\end{tabular}
}
\end{table}


Here are two more examples of tables:

\begin{table}[ht]
\caption{Comparison of Slow MPI Version and the Fast MPI Version}
\label{tbl:CSPV}
\centerline{
\begin{tabular}{|lc|c|c|c|c|c|c|c|c|}
\hline
\multicolumn{3}{|c|}{Parameters}   & \multicolumn{7}{|c|}{Process Number}\\ \hline
N &M & &1 &5 &10 &15 &20 &25 &30\\ \hline
128 &100 &Slow MPI(secs) &2.11 &3.91 &5.78 &8.26 &10.91 &14.17 &19.47\\ \cline {3-10}
    &    &Fast MPI(secs) &2.10 &1.20 &1.56 &1.95 &2.79 &3.22 &4.07\\ \hline
\end{tabular}
}
\end{table}


\begin{table}[h]
\caption{The Speedup of the MPI WRS Code and the HPF WRS Code}
\label{tbl:SPWC}
\centerline{
\begin{tabular}{|lc|c|c|c|c|c|c|c|c|}
\hline
\multicolumn{3}{|c|}{Parameters}   & \multicolumn{7}{|c|}{Process Number}\\ \hline
N &M & &1 &10 &20 &30 &40 &50 &60\\ \hline
128 &600 &MPI(speedup) &1 &5.18 &7.67 &8.24 &6.99 &5.55 &4.49\\ \cline {3-10}
    &    &HPF(speedup) &1 &8.40 &12.15 &13.98 &14.73 &13.52 &13.21\\ \hline
256 &300 &MPI(speedup) &1 &6.70 &7.74 &6.47 &5.19 &3.72 &2.94\\ \cline {3-10}
    &    &HPF(speedup) &0.99 &7.24 &9.65 &10.58 &10 &9.48 &8.73\\ \hline
512 &150 &MPI(speedup) &1 &6.75 &10.64 &12.14 &13.35 &13.87 &13.98\\ \cline {3-10}
    &    &HPF(speedup) &0.98 &6.72 &9.88 &11.55 &12.86 &13.38 &13.83\\ \hline
1024 &75 &MPI(speedup) &1 &2.13 &2.30 &2.36 &2.38 &2.39 &2.40\\ \cline {3-10}
    &    &HPF(speedup) &0.95 &1.94 &2.06 &2.10 &2.13 &2.13 &2.14\\ \hline
\end{tabular}
}
\end{table}

%%%%%%%%%%%%%%%%%%%%%%%%%%%%%%%%%%%%%%%%%%%%%%%%%%%%%%%%%%%%%%%%%%%%%%%%%%%%%%
%
% Chapter: Conclusions
%
%%%%%%%%%%%%%%%%%%%%%%%%%%%%%%%%%%%%%%%%%%%%%%%%%%%%%%%%%%%%%%%%%%%%%%%%%%%%%%

\chapter{Conclusions}


\section{What have we done so far?}

\section{Future directions}
The coming revolution in Wabi-sabi-lets offers many opportunities for further
research.

% The main text of the work ends here; the remaining text is back matter.
\backmatter

%%%%%%%%%%%%%%%%%%%%%%%%%%%%%%%%%%%%%%%%%%%%%%%%%%%%%%%%%%%%%%%%%%%%%%%%%%%%%%
%
% Bibilography
%
%%%%%%%%%%%%%%%%%%%%%%%%%%%%%%%%%%%%%%%%%%%%%%%%%%%%%%%%%%%%%%%%%%%%%%%%%%%%%%



% There are different bibliography styles; 'plain' is used for theses.
\bibliographystyle{plain}

% One way of constructing the bibliography is to list the entries explicitly
% in a 'thebibliography' environment, which is done here.
%
% If you prefer, you can use the 'bibtex' program, which formats the
% bibliography from a separate .bib file of "BibTeX" entries.  An advantage
% is that most references are available online in bibtex format, and
% doing cut-and-paste from the web browser can save you a lot of work.
% But it also adds extra complication.  The normal sequence, after you
% have changed anyting in the .bib file is
%
%   latex <filename>.tex
%   latex <filename>.tex
%   bibtex <filename>
%   latex <filename>.tex


% Literal Bibliography

% The 99 means make labels be as wide as the width of the number 99
\begin{thebibliography}{99}
\label{references}
\bibitem{ws:book1} Wabi-sabi The Great. \emph{The World According to Wabi-sabi.} NoWabi-sabi
Press, Wabi-sabiland, 1922.
\bibitem{ws:book2} Wabi-sabi The Great. \emph{The World Without Wabi-sabi.} NoWabi-sabi
Press, Wabi-sabiland, 1952.
\bibitem{me:paper1} Wabi-sabi Admirer. ``An Analysis of Wabi-sabiism.'' \emph{Journal of the
Advanced Computing Wabi-sabi}, vol. 21(2), pp. 272--287.
\bibitem{me:paper2} Wabi-sabi Admirer, Jim Smith and Jane Doe. ``Everyday
Wabi-sabiism.'' \emph{Journal of the Advanced Wabi-sabi Computing}, vol. 23(4), pp. 272--287.
\bibitem{guide} Graduate College. \emph{Standards for Preparation of Dissertations, Theses \&
Projects.} February 2021.
\bibitem{mainrepo} Amit Jain. \emph{Style files for PhD Dissertation or M.S. Thesis or Project}.\\
\texttt{https://github.com:/BoiseState/thesis-dissertation-template}
\bibitem{wabisabi} Wikipedia. \emph{Wabi-sabi}.\\
\texttt{http://en.wikipedia.org/wiki/Wabi-sabi}
\end{thebibliography}


%%%%%%%%%%%%%%%%%%%%%%%%%%%%%%%%%%%%%%%%%%%%%%%%%%%%%%%%%%%%%%%%%%%%%%%%%%%%%%
%
% Appendix
%
%%%%%%%%%%%%%%%%%%%%%%%%%%%%%%%%%%%%%%%%%%%%%%%%%%%%%%%%%%%%%%%%%%%%%%%%%%%%%%

% Use the \appendix command (only once) to start the appendix (or appendices)

\appendix

% After \appendix has executed, the \chapter command creates an appendix.
% Appendices are numbered by "A", "B", etc.

\chapter{Timing Measurements}\label{app:Timing}

Here is Appendix A. See Appendix~\ref{app:Setup} for the experimental setup.

% Test a figure in it to see if it works right

\chapter{Experimental Setup}\label{app:Setup}

Here is Appendix~\ref{app:Setup}.


% The \finish command is executed just before the \end{document}.
% Among other thigs, it produces the requisite blank page at the end
% of the document.
\finish  %! Do not remove!

\end{document}


